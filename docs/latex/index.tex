\label{index_md_Readme}%
\Hypertarget{index_md_Readme}%
 An A\+PI which provides implementation of various convex hull algorithms. Made as a part of the computation geometry course at B\+I\+TS Pilani, Hyderabad Campus

This A\+PI provides a \mbox{\hyperlink{class_point}{Point}} class with some general functions about it and 3 Convex Hull Algorithms\+:


\begin{DoxyItemize}
\item Graham\textquotesingle{}s Scan (\mbox{\hyperlink{graham_scan_8hpp}{graham\+Scan.\+hpp}} and \mbox{\hyperlink{graham_scan_8cpp}{graham\+Scan.\+cpp}})
\item Jarvis March (\mbox{\hyperlink{jarvis_march_8hpp}{jarvis\+March.\+hpp}} and \mbox{\hyperlink{jarvis_march_8cpp}{jarvis\+March.\+cpp}})
\item Quick\+Hull (\mbox{\hyperlink{quick_hull_8hpp}{quick\+Hull.\+hpp}} and \mbox{\hyperlink{quick_hull_8cpp}{quick\+Hull.\+cpp}})
\end{DoxyItemize}

Steps to use the A\+PI\+:


\begin{DoxyItemize}
\item Use \mbox{\hyperlink{utils_8hpp}{utils.\+hpp}} header for the \mbox{\hyperlink{class_point}{Point}} class
\item Include the header file algorithm\+Name.\+hpp ~\newline
 Eg\+: \#include $<$\mbox{\hyperlink{graham_scan_8hpp}{graham\+Scan.\+hpp}}$>$
\item Use the function for the respective algorithm
\item Compile your C++ file with the file algorithm\+Name.\+cpp and \mbox{\hyperlink{utils_8cpp}{utils.\+cpp}} ~\newline
 Eg\+: g++ test.\+cpp \mbox{\hyperlink{graham_scan_8cpp}{graham\+Scan.\+cpp}} \mbox{\hyperlink{utils_8cpp}{utils.\+cpp}} 
\end{DoxyItemize}